%%%%%%%%%%%%%%%%%%%%%%%%%%%%%%%%%%%%%%%%%%%%%%%%%%%%%%%%%%%%%%%%%%%%%%%%%%%%%%
\documentclass[12pt,hidelinks]{article}

% 1. Load LaTeX packages
\usepackage{fontspec}
\usepackage{geometry}
\usepackage{lastpage}
\usepackage{fancyhdr}
\usepackage{hyperref}
\usepackage{amsmath}
\usepackage{amsthm}
\usepackage{xunicode}
\usepackage{listings}
\usepackage{color}
\usepackage{amssymb}

% 2. Define page dimensions and spacing
\geometry{top=1in, bottom=1in, left=1in, right=2in, marginparsep=4pt,
          marginparwidth=1in}
\setlength{\parindent}{0pt}
\setlength{\parskip}{12pt}

% 3. Set header, footer, and bibliography
\renewcommand{\headrulewidth}{0pt}
\pagestyle{fancyplain}
\fancyhf{}
\lfoot{}
\rfoot{page \thepage\ of \pageref{LastPage}}
\bibliographystyle{acm}

% 4. Set fonts for the document
\defaultfontfeatures{Mapping=tex-text}
\setromanfont{YaleNew}

% 5. Define custom code for book environments and commands
\DeclareMathOperator*{\argmin}{arg\,min}
\DeclareMathOperator*{\argmax}{arg\,max}
\newcommand{\code}[1]{\texttt{#1}}
\newcommand{\pkg}[1]{\textbf{#1}}

% 6. Define custom code for book environments and commands
\definecolor{verbgray}{gray}{0.9}
\definecolor{verbgray2}{gray}{0.975}

\lstnewenvironment{rcode}{%
  \lstset{backgroundcolor=\color{verbgray},
  frame=single,
  framerule=0pt,
  basicstyle=\ttfamily,
  keepspaces=true,
  columns=fullflexible}}{}

\lstnewenvironment{rres}{%
  \lstset{backgroundcolor=\color{verbgray2},
  frame=single,
  framerule=0pt,
  basicstyle=\ttfamily,
  keepspaces=true,
  columns=fullflexible}}{}

% 7. Define numbering scheme for equations (only needed for handout).
\numberwithin{equation}{section}
\setcounter{section}{1}

%%%%%%%%%%%%%%%%%%%%%%%%%%%%%%%%%%%%%%%%%%%%%%%%%%%%%%%%%%%%%%%%%%%%%%%%%%%%%%
\begin{document}

{\LARGE Handout 03: Logistic Regression (Optional)}

\vspace*{18pt}


\textbf{Classification}

Many of the most interesting problems in statistical learning
require the prediction of a response variable that takes on
categorical values. Examples include spam detection, image
recognition, predicting the outcome of a sporting event, and
estimating the potential for a cataclysmic weather event. Our
methods have so far focused only on the estimation of a
continuous response variable. The flexibility of the linear
model, fortunately, allows us to extend it in ways
capable of estimating categorical responses.

To start, consider a task where the response variable takes
on only two categories. As a first step, we can create a numeric
response vector $Y$ the same way we constructed indicator
variables for the matrix $X$. For example, in a classifier to
predict the outcome of a sporting event we might define $Y$ as
\begin{align}
y_i &= \begin{cases} 1, & \text{category}_i =
   \text{win} \\ 0, & \text{category}_i \neq \text{win} \end{cases}.
\end{align}
Assuming we also have a model matrix $X$ there is no algorithmic
reason that we cannot directly apply linear regression using this
binary response $Y$.
How can we use the output of such a model to predict the classes
on a new dataset? The fitted values $\widehat{Y}$ will not be
exactly equal to $0$ or $1$. We therefore discretize the fitted
values according to
\begin{align}
z_i &= \begin{cases} 1, & \widehat{y_i} \geq 0.5 \\
 0, & \widehat{y_i} < 0.5 \end{cases} \label{disc_response}
\end{align}
and assign the predicted classes according the values $z_i$.
Splitting by the value of $0.5$ is reasonable being halfway
between $0$ and $1$, though of course it is possible to pick a
different cutoff value depending on the needs of a particular
problem.

Many of the nice theoretical properties of linear regression,
which unfortunately we have not been able to cover much due
to time constrains and the need for more probability theory,
are broken when we have discrete input data points. While the
strategy above can work well when needed, a modification based
on these theoretical properties often slightly outperforms the
naïve approach.

\textbf{Model of the log-odds ratio}

I want to start by defining the quantity $p_i$ to be the probability
that $y_i$ is equal to $1$. Otherwise, with probability $1-p_i$ the
variable $y_i$ will be equal to $0$. In the naive approach above
we are implicitly modeling:
\begin{align}
p_i &= x_i^t b = \sum_j x_{i, j} \cdot b_j.
\end{align}
One problem with this is that $x_i^t b$ can take on any real value
but the probability must be contained in the interval $[0,1]$. Instead,
what if we modelled the \textit{odds-ratio}. This is the probability of
observing a $1$ divided by the probability of observing a $0$ (gamblers
typically represent bets in terms of the odds-ratio). The odds-ratio can
take on any positive value, which is closer but still not the entire
real line. To fix this, we will model the \textit{logarithm} of the odds-ratio
as follows:
\begin{align}
\log\left(\frac{p_i}{1 - p_i}\right) &= x_i^t b.
\end{align}
Some arithmetic gets us a formula for $p_i$ directly:
\begin{align}
\frac{p_i}{1 - p_i} &= e^{x_i^t b} \\
p_i &= (1 - p_i) \cdot e^{x_i^t b} \\
p_i + p_i \cdot e^{x_i^t b} &= e^{x_i^t b} \\
p_i \cdot\left( 1 + e^{x_i^t b}\right) &= e^{x_i^t b} \\
p_i &= \frac{e^{x_i^t b}}{1 + e^{x_i^t b}}
\end{align}
It will also be nice to have a concise formula for $1-p_i$, the probability of
observing $y_i=0$:
\begin{align}
1 - p_i &= 1 - \frac{e^{x_i^t b}}{1 + e^{x_i^t b}} \\
&= \frac{1 + e^{x_i^t b}}{1 + e^{x_i^t b}} - \frac{e^{x_i^t b}}{1 + e^{x_i^t b}} \\
&= \frac{1}{1 + e^{x_i^t b}}
\end{align}

\textbf{Logistic regression}

Logistic regression uses the model defined above to take data vectors $x_i$
and produce predicted probability $p_i$ that given $y_i$ is equal to one. How
do we go about using training data to find the optimal values for the vector
$b$? The idea is to pick a vector $b$ that maximizes the probability
of observing our training data. If you have taken probability or statistics,
this is what you have probably heard called the maximum likelihood estimator
(MLE). How can we write down the probability of observing a certain set of
values? Notice that we can write the probability of observing a particular
$y_i$ in a clever way without the need for conditional logic (if this is
confusing, plug in $y_i=0$ and $y_i=1$ to see how this simplifies):
\begin{align}
Pr(y_i) &= p_i^{y_i} \cdot (1 - p_i)^{1 - y_i}.
\end{align}
The probability of observing two independent events is the product of their
probabilities. Assuming the data observations are independent, this gives us:
\begin{align}
Pr(y) &= \prod_i p_i^{y_i} \cdot (1 - p_i)^{1 - y_i}. \label{peq}
\end{align}
We want to find a value for $b$ that maximizes this quantity, but note
that maximizing a quantity is the same as maximizing the logarithm of a quantity
(the \textit{maximum} itself is different, but the value that maximizes it is
the same). Therefore we will use the negative logarithm of the value in
Equation~\ref{peq} as our loss function:
\begin{align}
l(b) &= - \left( \sum_i y_i \cdot \log(p_i) + (1 - y_i) \cdot \log(1 - p_i) \right) \\
&= - \left( \sum_i \log(1 - p_i) + y_i \cdot \log\left[ \frac{p_i}{1 - p_i} \right] \right)
\end{align}
Here we can plug in our formula for $1-p_i$ and $p_i$:
\begin{align}
l(b) &= - \left( \sum_i \log\left[ \frac{1}{1 + e^{x_i^t b}} \right]
+ y_i \cdot x_i^t b \right) \\
&= \sum_i \log(1 + e^{x_i^t b}) - y_i x_i^t b.
\end{align}
Now, we want to minimize the quantity. What are the partial derivatives with
respect to $b_j$?
\begin{align}
\frac{\partial l(b)}{\partial b_j} &= \sum_i \frac{1}{1 + e^{x_i^t b}} \cdot
e^{x_i^t b} \cdot x_{i,j} - y_i x_{i, j} \\
&= \sum_i \left( p_i \cdot x_{i,j} - y_i x_{i, j} \right) \\
&= \sum_i \left( p_i - y_i \right) \cdot x_{i,j}
\end{align}
Writing this as a matrix equation, we can write down the entire gradient
compactly as:
\begin{align}
\nabla_b l(b) &= X^t (p - y).
\end{align}
The gradient with respect to $\beta$ of the log-likelihood is known as
the \textit{score function}.
To find the maximum likelihood estimators we
need to find where the gradient is equal to zero. Notice that at the minimizer
the residuals $y - p$ are perpendicular to the column space of $X$. This is
exactly what you showed held for the ordinary least squares solution as
well!

Finally, note that if we take following two formulas:
\begin{align*}
p_i &= \frac{e^{x_i^t b}}{e^{x_i^t b} + 1} \\
1 - p_i &= \frac{1}{1 + e^{x_i^t b}}
\end{align*}
And multiple both of these by:
\begin{align*}
1 &= \frac{e^{-x_i^t b}}{e^{-x_i^t b}}
\end{align*}
You will get the alternative forms:
\begin{align*}
p_i &= \frac{1}{1 + e^{-x_i^t b}} \\
1 - p_i &= \frac{e^{-x_i^t b}}{e^{-x_i^t b} + 1}
\end{align*}
So if you multiply $x_i^t b$ by negative one, you flip the equations for $p_i$
and $(1 - p_i)$. So, logistic regression is in some sense \textit{symmetric}
between $0$ and $1$. If you flip the signs of the regression vector $\beta$
you will have the logistic model with the two classes flipped.

\textbf{Solving the logistic score function with gradient descent}

Unfortunately we cannot write down an analytic solution that sets the
logistic score function to zero. We need an iterative way of finding a solution
where the gradient is zero (or, at least, sufficiently small). The way that
most statistical programs, such as R and Python, solve the logistic regression
equation is using a second-order method known as Newton-Ralphson updating. In
this context it is also sometimes called Iteratively Reweighted Least Squares
(IRWLS). We are instead going to look at a different approach; it is not as
good as IRWLS for this specific problem, but can be adapted far more readily
to other estimators we will see in this course.

Gradient descent is an algorithm that tries to minimize a function by taking
small steps in the direction of the negative gradient. Lets say that we start
with a guess for the regression vector $b^{(0)}$. Gradient descent would then
move to a new estimator defined by:
\begin{align}
b^{(1)} &= b^{(0)} - \rho \cdot \nabla_b l(b^{(0)}).
\end{align}
The same equation is used to move from $b^{(1)}$ to $b^{(2)}$, and so forth.
In theory, it moves in the direction that $l(b)$ is decreasing the fastest. We
can iterate this until either the values of $b$ stop changing very much or the
gradient is sufficiently small. The value $\rho$ is a positive constant called
the \textit{learning rate}. Too large and the iteration will jump wildly over
the input space and never converge on a good value for $b$. Too small and it
will take forever to actually reach (if ever) a good value. You need to
manually adjust $\rho$ in order to find a good balance.


\end{document}
